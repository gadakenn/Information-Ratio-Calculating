\documentclass[14pt]{extarticle}


\usepackage[english, russian]{babel}



% Useful packages
\usepackage{amsmath}
\usepackage{graphicx}
% \usepackage{tempora} %Times New Roman alike
\usepackage[colorlinks=true, allcolors=blue]{hyperref}
\usepackage{times}

\usepackage[a4paper, margin=1in]{geometry} 
\usepackage{setspace}
\onehalfspacing 

\title{Обзор литературы по теме "Исследования фондовых рынков"}
\author{Комаров Никита, БЭАД223}
\date{}

\begin{document}
\maketitle

\begin{abstract}
Фондовые рынки всегда были предметом интенсивных исследований в области финансов и экономики. Они привлекают внимание как практикующих специалистов и инвесторов, так и академического сообщества. В последние годы интерес к этой теме значительно возрос, особенно в контексте глобализации, технологических инноваций и изменяющихся регуляторных рамок. Более того, общество всё становится всё более и более образованным, а, значит, задумывается о возможных путях сохранения или приумножения капитала. Важность этих рынков в мировой экономике невозможно переоценить, поскольку они не только отражают текущее состояние экономики, но и служат важным индикатором будущих тенденций и возможностей. В обзоре литературы будут рассмотрены современные научные статьи, которые в той или иной мере связаны с исследованием фондовых рынков. 
\end{abstract}

\section{Основная часть}


В последние десятилетия фондовые рынки заняли ключевое место в мировой экономике. Особый интерес вызывает российский фондовый рынок, который несмотря на относительную молодость, играет значительную роль в мировой финансовой системе. Его развитие началось в 1990-х и активно продолжается по сей день.
Статья Полозовой \cite{1} поднимает вопросы, связанные с текущим состоянием и перспективами этого рынка, выделяя его как один из самых прибыльных и обладающих значительным потенциалом среди мировых рынков.

Рост индекса Московской биржи, который в 2019 году превысил 3000 пунктов, демонстрирует внушительные 349\% роста в долларовом выражении. Это, по мнению автора, отражает не только уверенность, но и недооцененность российских компаний, а также государственные инициативы по увеличению дивидендных выплат, способствующие притоку инвесторов. Такие меры, безусловно, придают рынку динамику, однако при этом автор не уделяет должного внимания потенциальным рискам, таким как политическая нестабильность, санкции и волатильность цен на нефть, которые могут влиять на стабильность роста.

Исследование Полозовой подчеркивает, что несмотря на определенные проблемы, российский фондовый рынок остается одним из наиболее привлекательных для инвестиций благодаря высоким дивидендам и положительной динамике в экспорте. Однако важно критически оценивать эти данные, не забывая о возможных рисках и неопределенностях, которые также являются частью инвестиционной среды. Автор поправку на этот счёт не делает.

В 2010-х сущствовало стремление к интеграции российского фондового рынка с мировыми финансовыми системами, как со стороны российского правительства, так и со стороны многих участников рынка. Многие государства (и сама Россия в том числе) видели в глобализации возможность привлечении инвестиций со стороны, а также увеличение конкурентноспособности российских компаний. Однако ввиду политических напряжённостей, уязвимости России для колебаний цен на нефть, глобальной пандемии COVID-19 интеграция национального фондового рынка ограничилась и, можно сказать, изменила вектор своего "вхождения" в обратную сторону.

Продолжая наш обзор, статья Е.В. Семенковой и Е.Н. Власовой \cite{2} представляет собой многогранное исследование, которое затрагивает ключевые аспекты влияния глобализации и деглобализации на российскую экономику и фондовый рынок.

В оценке исследования необходимо отметить, что хотя авторы и подчеркивают позитивное влияние глобализации на нефтегазовый сектор и его корреляцию с горнодобывающим сектором, они в то же время выявляют значительные негативные эффекты для металлургической отрасли, которая столкнулась с усиленной конкуренцией импорта. Это подчеркивает, что глобализация может быть процессом, который может иметь двойственное влияние: способен одновременно стимулировать и подавлять развитие определенных отраслей.

Кроме того, статья указывает на общую тенденцию к деглобализации, которая, однако, оказывает ограниченное влияние на финансовый сектор и рынок ценных бумаг. В этом контексте, хотя авторы обращают внимание на сохранение финансовых потоков и возможности для российских инвесторов, они не углубляются в анализ потенциальных долгосрочных последствий деглобализации для этих финансовых потоков и степени интеграции российского рынка в мировую финансовую систему.

Отдельное внимание в статье уделяется последствиям пандемии COVID-19, которая ускорила процессы деглобализации. Авторы правильно указывают на COVID-19 как на фундаментальный фактор, усиливающий деглобализационные тенденции. Однако они могли бы также рассмотреть будущее влияние пандемии на изменения моделей глобальной торговли и инвестиций и то, как эти изменения повлияли бы на российский фондовый рынок.

Говоря более подробно про COVID-19 стоит отметить крах фондового рынка в 2020. Американский фондовый рынок в момент обрушения рассматривают М. Мазур, М. Данг и М. Вега \cite{MAZUR2021101690}.
Исследование показывает, что последствия краха были неравномерными по разным секторам. Авторы обнаружили, что акции в секторах природного газа, пищевой промышленности, здравоохранения и программного обеспечения показали высокую доходность, в то время как стоимости акций компаний из секторов нефти, недвижимости, развлечений резко упали. В статье хорошо объясняются причины такого обвала.


Статья также освещает разнообразные реакции компаний на шок доходов во время COVID-19. Она показывает, что, хотя некоторые компании снижали зарплаты топ-менеджерам и дивиденды, в то время как удивительно утверждали новые бонусы и повышения зарплат, что может указывать на плохое корпоративное управление. Такое поведение наводит на мысль, что некоторые компании могут использовать ситуацию в ущерб акционерам, что поднимает вопросы об эффективности их управленческих структур.

Тем не менее, исследование имеет некоторые недочёты. Несмотря на то, что используемый набор данных достаточно обширный, статья сосредоточена на непосредственных последствиях краха и не охватывает возможные долгосрочные последствия пандемии. Кроме того, использование вручную собранных данных из отчетов 8K и DEF14A, доступных на SEC EDGAR\footnote{SEC EDGAR - система, используемая Комиссией по ценным бумагам и биржам США (SEC) для автоматической сборки, анализа и распространения финансовой информации.}, может привести к потенциальному смещению выборки, поскольку не все реакции компаний могли быть зафиксированы или могли быть раскрыты после периода исследования.

Статья открывает пути для дальнейших исследований, особенно в области корпоративного управления и его влияния на поведение фирм в кризисные периоды.

Времена COVID-19 осложнили деятельность не только американских компаний. Влияние пандемии на весь мировой рынок также было значительным. Исследованием этого вопроса занимались многие профессиональные экономисты и трейдеры, в их числе Зекай Ш. и
Фейяз З.\cite{article_721871} Авторы исследуют мирововые фондовые рынки с января по апрель 2020 года, используя методологию Фурье-Коинтеграции\footnote{Методология Фурье-Коинтеграции — статистический подход, используемый для анализа долгосрочных взаимосвязей между временными рядами. Позволяет обнаруживать и учитывать периодические колебания и нелинейные тренды в данных.}, и показывают, что пандемия вызвала резкое падение большинства индексов и значительные убытки для инвесторов (снижение стоимости финансовых инструментов). Они также подчеркивают, что воздействие COVID-19 на мировую экономику и фондовые рынки зависит от множества факторов, включая скорость распространения вируса, разработку вакцин и лекарств, а также экономические и финансовые меры, предпринимаемые правительствами.

Как известно, в современном мире происходило немало других пандемий, поэтому в статье авторы также рассматривают SARS, Ebola и H7N9, которые также имели отрицательное влияние на фондовые рынки. Сходство результатов этого исследования с предыдущим изучением других  подчеркивает долгосрочную связь между эпидемиями и фондовыми рынками.

Хорошо структурированная информация легко воспринимается даже людьми без должного экономического образования, что, конечно же, является плюсом для авторов.

Что же можно сказать про политические новости и, как следствие б\'{о}льшую волатильность рынка? Инвесторы часто быстро реагируют на возможную неопределенность и напряженность бизнес-климата. Это особенно заметно в случаях значительных политических событий, таких как выборы, изменения в государственной политике или международные конфликты, которые могут влиять на экономические перспективы и стабильность рынков. В своей статье "Политические новости и волатильность фондового рынка"\cite{NBERw25720}\ Скотт Р. Бейкер, Николас Блум, Стивен Дж. Дэвис и Кайл Дж. Кост изучают влияние новостей о политике на волатильность фондовых рынков. Авторы создают свой трекер волатильности фондового рынка (EMV), основанный на новостных статьях, и анализирует, как различные категории новостей, особенно политических, влияют на волатильность. Основное внимание уделяется новостям о макроэкономической перспективе, коммодити\footnote{Коммодити (иногда биржевой товар или коммодитиз от англ. commodity) — товары, активно перепродаваемые на организованных рынках. }, фискальной\footnote{Фискальная политика — правительственная политика, представляющая собой меры воздействия на экономику с помощью изменения величины расходов или доходов государственного бюджета.} и монетарной политике\footnote{Монетарная политика – это процесс, с помощью которого центральный банк страны управляет денежно-кредитными условиями, влияя на ставки процента и уровень денежной массы, чтобы контролировать инфляцию и стабилизировать валюту, а также для достижения экономических целей.}.
Результаты показывают, что различные новостные категории оказывают различное влияние на волатильность рынка. Политические новости, особенно о макроэкономической политике, играют значительную роль в колебаниях рынка. Эти результаты сравниваются с ранее проведенными исследованиями, которые также подтверждают влияние политических и экономических новостей на фондовые рынки. Преимуществом данного исследования является его масштабируемость и способность адаптироваться к разным странам и временным периодам. Однако методология, основанная на частотном анализе статей, может не учитывать всех потенциально значимых новостей или быть смещенной в сторону определенных тематик. Исследование основывается на предположении, что частота упоминаний определенных тем в новостях коррелирует с волатильностью рынка. При этом предполагается, что упоминания этих тем в новостях являются индикаторами рыночной неопределенности или важных экономических событий, влияющих на рынки. Это допущение может не всегда точно отражать реальное влияние новостей на рынок, так как не учитывает контекст или тон самой новости.

Интересно посмотреть на то, как можно в числовом эквиваленте оценить неопределенность в мире. Этим вопросом занимались Хайтс Ахир, Николас Блум и Дэвид Фурсери\cite{NBERw29763}. В статье они представляют индекс неопределенности (WUI), оценивающий частоту упоминаний слова "неопределенность" в докладах Economist Intelligence Unit (EIU) с 1952 года по 143 странам. Индекс фокусируется на крупных событиях, таких как войны и экономические кризисы, и выявляет их влияние на экономическую активность, особенно в странах с низким уровнем институционального качества. Результаты показывают, что уровень неопределенности выше в развивающихся странах и более синхронизирован в развитых экономиках. Анализ подтверждает, что неопределенность предвещает существенное снижение производства, что особенно заметно в странах с низким уровнем институционального качества. Эти выводы согласуются с теоретическими работами по влиянию экономической неопределенности на рост. Исследование предполагает, что частота упоминаний "неопределенности" в EIU отчетах отражает уровень глобальной экономической неопределенности. При этом предполагается, что эти упоминания коррелируют с важными экономическими и политическими событиями, влияющими на рыночные условия и экономическую активность в разных странах.

Говоря о влиянии на фондовый рынок более подробно, стоит отметить, что банковский сектор является важным компонентом всей мировой экономики. Это означает, что банкротство какого-либо крупного банка может иметь долгосрочные последствия. Банкротство Silikon Valley Bank (далее - SVB)\cite{PANDEY2023104013} изучили Дарен Пандей, М. Кабир Хассан и Рашедул Хасан. Банкротство SVB вызвало панику и неопределенность, что привело к значительному падению доходности на рынках по всему миру. Авторы использовали ивент-анализ для изучения аномальной доходности рынков в результате банкротства. В статье подчеркивается, что банкротство банка значительно влияет как на отечественные, так и на международные фондовые рынки. Результаты показывают, что развитые рынки были более уязвимыми по сравнению с развивающимися рынками. Преимуществом данного исследования является использование всестороннего подхода ESM \footnote{Европейский стабилизационный механизм (European Stability Mechanism) — постоянно действующий фонд финансовой стабилизации стран Еврозоны, заменяющий существующие два фонда: Европейский фонд финансовой стабильности, располагающий гарантированными странами Еврозоны 440 миллиардами евро, и Европейский механизм финансовой стабилизации.}.
Результаты исследования демонстрируют существенное негативное воздействие банкротства SVB на глобальные рынки, особенно в развитых экономиках, и выявляют значительные различия в воздействии на рынки различных стран и регионов. Было отмечено, что развитые рынки более подвержены внешнему шоку (такому как банкротство SVB) из-за более высокого уровня интеграции и взаимозависимости с глобальной экономикой. В послесобытийный период была отмечена значительная анормальная волатильность, особенно на развитых рынках.

Анализ фондовых рынков заключается не только оценке влияния событий на те или иные индексы. Довольно интересной является тема изучения стратегий инвестирования. В своей статье Джейн Манси, Шарма Гаган и Сривастава Мриналини\cite{risks7010015} размышляют, могут ли устойчивые инвестиционные альтернативы предложить лучшие финансовые результаты, чем традиционные индексы (бенчмарки) из развитых и развивающихся рынков. Основная цель исследования - сравнить финансовые доходы от глобальных индексов S-Network, включая индексы ESG развитых и развивающихся рынков, с традиционными рыночными индексами, такими как индексы MSCI. Авторы также сосредоточились на взаимосвязях и перетоках волатильности между этими индексами. В исследовании применяются различные методы: корреляционный анализ, моделирование ARCH\footnote{Авторегрессионная условная гетероскедастичность (англ. ARCH — AutoRegressive Conditional Heteroscedasticity) — применяемая в эконометрике модель для анализа временных рядов (в первую очередь финансовых), у которых условная (по прошлым значениям ряда) дисперсия ряда зависит от прошлых значений ряда, прошлых значений этих дисперсий и иных факторов.}-GARCH и другие эконометрические модели. Это позволяет авторам изучить как взаимосвязь между доходностью устойчивых и традиционных индексов, так и переток волатильности между ними.

Анализируя статью стоит отметить, что между устойчивыми индексами и традиционными индексами MSCI существует интеграция и информационный поток, но в целом разница в производительности между ними не значительна. Это даёт читателю понять, что устойчивые инвестиции могут быть эффективной альтернативой традиционным инвестициям с точки зрения доходности.

Таким образом, благодаря чёткости высказываний и хорошо сформулированным тезисам, читателю статьи достаточно несложно понять суть альтернативных инвестиций. Так как методы действительно взаимозаменяют друг друга, то читающий энтузиаст имеет возможность попробовать метод, изученный авторами.

Также многие инвесторы занимаются своей деятельностью преимущественно ради получения дивидендов. Следовательно, их интересуют стратегии с более прогнозируемым результатом. Статья Сэмюэла М. Харцмарка и Дэвида Х. Соломона "The Dividend Disconnect"\cite{2fef20e6-54c7-3c5b-8792-6382b832ccff}исследует, как инвесторы воспринимают дивиденды и капитальные приросты как отдельные атрибуты, не осознавая, что выплата дивидендов приводит к уменьшению цены акций. В работе рассматривается, как это влияет на различные аспекты торговли, включая оценку прибылей и убытков, ценообразование акций, прогнозы аналитиков и реинвестирование дивидендов. Авторы отмечают, что традиционные предположения в академической финансовой сфере о том, что инвесторы равнодушны между получением дивидендов и реализацией акций, не всегда совпадают с реальным поведением инвесторов. Представленные данные подчеркивают, что инвесторы считают дивиденды и капитальные приросты раздельными источниками дохода, что влияет на их торговое поведение и оценку акций. Авторы также выдвигают предложение о действии эффекта диспозиции\footnote{Эффект диспозиции (предрасположенности) — это склонность инвесторов продавать прибыльные акции, а убыточные активы, наоборот, держать в портфеле в надежде на то, что они «отрастут».}, при котором инвесторы склонны продавать акции, показывающие прибыль, а не убыток.
В статье авторы подробно анализируют как дивидендов и капитальных приростов влияет на поведение инвесторов, что имеет практическое значение для понимания динамики фондового рынка новоиспеченными инвесторами. С другой стороны стоит отметить, что Харцмарк и Соломон не рассматривают возможные позитивные аспекты раздельного восприятия дивидендов и капитальных приростов.

Сегодня очень популярной является разработка моделей для предсказания роста или падения стоимости той или иной ценной бумаги, некоторые строят модели, которые анализируют новости и прогнозируют состояние портфеля. К примеру, Р. Рену Изидор and П. Кристи\cite{7024cc3b-10df-38e0-a92c-a238844fe85c} в своей статье исследуют влияние поведенческих предубеждений инвесторов на фактические доходы от инвестиций в акции на вторичном рынке. Целью исследования было разработать регрессионную модель, предсказывающую фактический доход от инвестиций, используя поведенческие предубеждения в качестве предикторов. Исследование было проведено среди 436 инвесторов в Ченнаи и охватывает восемь поведенческих предубеждений: репрезентативность, чрезмерную самоуверенность,
привязанность, заблуждение игрока, предвзятость доступности, непринятие потерь, неприятие сожалений, ментальный учет и предвзятость оптимизма. Целью исследования было разработать регрессионную модель, предсказывающую фактический доход от инвестиций, используя поведенческие предубеждения в качестве предикторов. Изучение данной тематики основано на предположении, что поведенческие предубеждения могут значительно влиять на фактические доходы от инвестиций. Авторы предполагают, что уровень каждого из предубеждений коррелирует с финансовыми результатами инвесторов, влияя как на рациональное принятие решений, так и на финансовые исходы.
Полезность для читателя (в частности для финансовых консультантов и инвесторов) заключается в том, что статья выявляет связь между поведенческими предубеждениями и финансовыми результатами инвесторов. Однако исследование ограничено выборкой из одного города и может не полностью отражать поведение инвесторов в других регионах или странах. В то же время некоторые предубеждения, такие как якорение\footnote{Якорение в психологии и принятии решений - это когнитивная предвзятость, при которой люди слишком сильно полагаются на первоначальную информацию («якорь») при принятии решений.}, самоуверенность и оптимизм, показали положительное влияние на доходы от инвестиций. 

Многие модели также строятся на оценке влияния новостей на фондовый рынок. Они обучаются на миллионах заголовках и ещё б\'{о}льшем количестве соответствующих временному промежутку котировок. Этим как раззанялись Яньхун Се, Хунцзюнь Цзян \cite{Xie_2017} о чём они и говорят в своей статье "Stock Market Forecasting Based on Text Mining Technology: A Support Vector Machine Method". В исследовании собраны 2302692 новостных материалов с 2008 по 2015 год. Авторы формируют специализированный словарь стоп-слов и точный словарь настроений, а затем предлагают прогнозирующую модель на основе машины опорных векторов. Большим плюсом данного исследования является использование инновационных методов текстового анализа и анализа настроений для прогнозирования рыночных тенденций. Хочется отметить, что, к примеру, студенту моего направления эта информация очень полезна, поскольку построение предсказывающих моделей - это одна из основных задач машинного и глубинного обучения, которое будет изучаться на нашей программе в последующих курсах. Единственным минусом является фокусировка только на китайском рынке и использование данных только за определенный период, что может не отражать долгосрочные тенденции. Результаты исследования показывают, что новости оказывают значительное влияние на фондовые рынки. Авторы обнаружили, что расширение вектора входных данных для дополнительных ситуаций, когда в тот день нет новостных данных, эффективнее в моделях SVR, но хуже в моделях SVC. Они также выявили, что время влияния новостей на фондовый рынок составляет менее двух дней, что является довольно неочевидным выводом.

Так как тема машинного обучения и исследования фондовых рынков с каждым днём всё более и более близкие области, то и методов для обучения моделей становится всё больше. В своей статье\cite{upadhyaytechnical} У. Ишитва и К. Девашиш и В. Васудха исследуют применение инструментов Data Science, Machine Learning (ML) и Artificial Intelligence (AI) для прогнозирования движений акций и рыночных тенденций. Основное внимание уделяется применению трех методов ML — деревьев решений, искусственных нейронных сетей, а также роли временных рядов и обработки естественного языка (NLP - Natural Language Processing) в прогнозировании цен на акции. В работе авторы выдвигают предположение о том, что современные методы ML и NLP могут эффективно анализировать и прогнозировать движение рынка на основе как исторических данных, так и текущего общественного мнения, выраженного через новостные статьи и социальные сети. И у них правда это получается, поскольку в результате проведенного исследования обученные модели ANN и SVM достигают высокой точности до 99,9\% в прогнозировании рынка. В то же время авторы указывают на сложность и неструктурированность финансовых данных, что делает прогнозирование более сложной задачей. Это говорит нам о том, что данные методы, конечно, имеют высокую доходность в применении, однако требуют большого количества ресурсов для первоначальной подготовки (обучения модели).

\section{Заключение}

Таким образом, были рассмотрены статьи, в которых авторы исследуют фондовые рынки и их зависимости с остальным миром. Более того, в анализы вошли статьи, в которых исследуются фондовые рынки с использованием современных технологий анализа больших данных. Хочется отметить, что именно эта тема является самой близкой к моей будущей деятельности, поэтому по мере обучения и изучения строения фондовых рынков и дизайна экономических механизмов знания по данному вопросу будут расширятся. Этот обзор литературы будет полезен людям, интересующимся фондовыми рынками, трейдингом и инвестированием.
\bibliographystyle{plain}
\bibliography{sample}

\end{document}